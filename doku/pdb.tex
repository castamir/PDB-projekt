\documentclass[11pt,a4paper]{article}
\input{config}
\usepackage{amsmath}
\usepackage{moreverb}


\begin{document} \sloppy
\titlepageandcontents

\section{Spuštení a nastavení aplikace}
Pro spuštení projektu stačí otevřít obsah rozbaleného archivu jako projekt ve vývojovém studiu \emph{NetBeans}\footnote{\url{https://netbeans.org/}}.

\section{Přihlašovací obrazovka a naplnění vzorové databáze}\label{sec1}
Po spuštění aplikace se jako první záložka aplikace objeví administrace. Na této kartě je možné:
\begin{enumerate}
 \item přihlásit se -- po vyplění údajů a stisku tlačítka 1 z obrázku \ref{admin}
 \item přihlásit se jako host -- v případě, že máme dostupný konfigurační soubor (více viz kapitola \ref{conf}), a stiskem tlačítka 2
 \item resetovat databázi do původního stavu -- stiskem tlačítka 3
 \end{enumerate}
 Možnost resetovat databázi je dostupná až po přihlášení uživatele.

%---obrazek
\begin{figure}[h!]
\begin{center}
\scalebox{1.0}{\includegraphics{images/login.png}}
\caption{Obrazovka administrace.}
\label{admin}
\end{center}
\end{figure}
%---konec obrazek

 \subsection{Konfigurační soubor}\label{conf}
 Aby se nemuseli pro každé přihlášení znovu vyplňovat přihlašovací údaje, je v aplikaci možnost automatického připojení skrze údaje uložené v konfiguračním souboru. Ten je ve tvaru:
 \begin{center}
\begin{boxedverbatim}
DB.LOGIN=login
DB.PASSWORD=password

DB.HOST=hostaddress
DB.PORT=port
DB.SID=dbname
\end{boxedverbatim}
\end{center}
Tento soubor se musí jmenovat \texttt{config.local.properties} a musí být v adresáři \verb;PDB-projekt\src\cz\vutbr\fit\pdb\config;. V případě existence tohoto souboru se pokusí aplikace po spuštení připojit k databázi automaticky, aniž by bylo nutné na něco klikat.

\section{Ovládání aplikace}
Aplikace je rozdělena do několika záložek: 
\begin{itemize}
\item \textbf{Administrace} -- popsána v sekci \ref{sec1}
\item \textbf{Služby} -- V této záložce je zobrazena mapa areálu. Při kliknutí (pokud je aktivní zezelená) na objekt je možné tento objekt zarezervovat pro ubytované hosty. Rezervuje se na hodinu a je možné zvolit i konkrétní datum. Rezervace se potvrdí stiskem tlačítka ``Uložit změny''.
\item \textbf{Areál} -- V této záložce je možné areál upravovat a provádět operace nad prostorovými daty.
\item \textbf{Fotografie} -- V této záložce je možné si prohlédnout uložené fotografie automobilů zákazníků a případně vyhledávat podle obsahu fotografie. Dále je zde možné s fotografiemi otáčet a mazat je. Pro jakoukoliv operaci (vyjma přechodu na předchozí/další fotografii) s fotografiemi je ji nutné nejdříve označit levým klikem na fotografii (projeví se zeleným orámováním fotografie).
\item \textbf{Rezervace} -- V této záložce je možné si zobrazit seznam rezervací pokojů v zadaném období, dále je možné vytvořit novou rezervaci a případně vyhlásit stav karantény.
\end{itemize}
%---obrazek
\begin{figure}[h!]
\begin{center}
\scalebox{0.5}{\includegraphics{images/sluzby.png}}
\caption{Obrazovka služeb.}
\label{sluzby}
\end{center}
\end{figure}
%---konec obrazek
\end{document}
